\subsection{Mark-Recapture Estimation}

Mark-recapture analysis (or capture-recapture analysis) is a widely-used and well-developed method to estimate sizes of biological populations \citep{amstrup2010handbook}.
This method uses the proportion of individuals shared between two or more samples as a proxy to estimate the total population size that is being sampled.
For large population sizes, relatively lower recapture rates are expected.
It turns out, though, that idealized sampling from an urn poorly describes animal behavior.
Various potential biases have been identified --- ranging from the inherent disposition of certain animals to be more ``trap happy'' or ``trap shy'' to the tendency of already-captured animals to become more wary of traps --- and sophisticated statistical methods have been devised to make estimation robust to them.
Mark-recapture literature, therefore, provides a rich, ready-made buffet for tacking estimation problems involving repeat partial sampling.

In one application, \citet{moreno2024methods} demonstrate use of a mark-recapture estimator in quantifying sites contributing to fitness that are not individually detectable due to epistatic redundancy and or small-effect contributions.
Analogy to the mark-recapture scenario is established by equating sites with any potential for fitness effect --- whether or not detectable through single-site knockout --- to the population to be estimated.
Iterative knockouts are applied to produce several ``skeleton'' genotypes, where no more sites can be removed without reducing fitness.
Sites in each skeleton, therefore, each have a demonstrable fitness effect --- but, if redundancy or small effects are at play, no skeleton contains all such sites.
Each skeleton, therefore, represents a sample of sites with potential fitness effects.
Crucially, though, these samples will overrepresent lower-redundancy or larger-effect sites.
Application of a jackknife estimator due to \citet{burnham1979robust}, however, ensures estimation accuracy remains intact.
In a separate line of work, \citet{schulte2014software} have noted potential for mark-recapture methods to play a role in characterizing the extent of neutral space within multistep mutational neighborhoods of computer programs.
