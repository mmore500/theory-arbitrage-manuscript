% ELD: I see what you mean about the framing being tricky. Currently the flow of the intro seems to be:
% - The framing of EC has historically been very biosinspired
% - This is good because of common vocab and charisma
% - But maybe its bad because it leads us to implement algorithms purely on the basis of the naturalistic fallacy
% - This is bad because it hurts EC's credibility
% - We could fix that problem by abandoning bioinspiration, but we're going to aruge that its actually better to lean further into bioinspiration and use it to improve the theoretical backing of EC
% - Further set-up and elaboration on this theory
%
% I think on first reading it currently leans a little further than ideal into sounding like we're going to argue that the bio-inspiration is bad. Which might alienate the readers we are most interested in attracting (then again, maybe it would attract the readers we most want to change the minds of). It would probably help to be clearer about what we think and what others think. Going to try to make some edits to this effect.

\section{Introduction} \label{sec:introduction}

The metaphor of biological evolution so deeply pervades evolutionary computation (EC) that difficulty can arise in discerning the threshold where bio-inspiration ends and algorithm engineering begins.
Although some work delves well outside the biological metaphor \citep{hansen2001completely,munteanu1999improving,miller2015cartesian}, core aspects of EC prevail within a bio-inspired framing --- whether originally conceived as such, or subsequently co-opted to an analog in nature.
Table \ref{tab:bioinspiration} highlights prominent examples of this phenomenon.

\begin{table*}
\centering
\caption{Conceptual analogies between evolutionary computation (EC) mechanisms and biological processes.}
\label{tab:bioinspiration}
\renewcommand{\arraystretch}{1.6} % vertical padding between table rows
\begin{tabular}{@{}p{4.2cm}p{4.2cm}p{6.5cm}@{}}
\toprule
\textbf{EC Mechanism} & \textbf{Biological Analogy} & \textbf{References} \\
\midrule
Diversity maintenance & Negative frequency--dependent selection (ecology) & \citep{dolson2018ecological,dolson2018applying} \\
Reciprocal selection & Co-evolution & \citep{lehman2010efficiently,harper2012spatial,garbus2024accelerating,koza1991genetic,kala2012multi,wang2019poet,miikkulainen2024evolving} \\
Inexact referencing (e.g.\ tags) & Active-site recognition in biomolecular interactions & \citep{spector2011tag,moreno2023matchmaker,lalejini2018signalgp,downing2015intelligence} \\
Search-space transforms & Genotype-phenotype maps & \citep{lehman2023evolution,moreno2018learning,bentley2022evolving,gaier2020discovering,wittenberg2023denoising} TODO \\
\bottomrule
\end{tabular}
\end{table*}


Indeed, given the prevalence of bio-inspiration within evolutionary computing (EC), the merit of bio-inspiration has constituted a longstanding point of discussion.
In addition to imparting useful new ideas, some highlight value from biological metaphors simply as a source for convenient, intuitive, self-consistent vocabulary \citep{sorensen2015metaheuristics,banzhaf2006artificial}.
Another draw of bio-inspiration is the charisma it lends evolutionary computation \citep{lehman2020surprising}.
The complexity and emergent intelligence found throughout the natural world provides a stunning display of evolution’s profound creative power, and the prospect of capturing any fraction of that power and creativity is fascinating.
% Could also mention that a lot of people find evolution more intuitive
%The peculiar ubiquity of bio-inspiration in evolutionary computation compared to other ML/AI domains, however,
Indeed, the demonstrated viability of natural precedent lends bio-inspiration credence as a blueprint for ambitious research agendas in computing \citep{miikkulainen2021biological,banzhaf2006from}.

Some have questioned, however, whether charisma of the evolutionary approach might also be serving as a preoccupying distraction \citep{moore2023evolution,sorensen2015metaheuristics}.
In particular, given the broader tendency for machine learning researchers to adhere to their preferred models \citep{domingos2012few}, there is risk that evolutionary approaches have become something of a pursuit for its own sake, rather than a means to an end \citep{woodward2016gp,yampolskiy2018we}.

% ELD: I like this, but I think it's breaking the flow. Maybe it can go somewhere else
%Bio-inspiration, additionally, can be perceived to play into criticism of the \textit{ad hoc} structure of evolutionary computation.
%As typical in outcome-driven AI/ML fields, substantial portions of the EC literature pursue descriptive rather than explanatory objectives --- i.e., ``we found that $X$ approach benefited solution quality'' as opposed to ``we evaluated how $Y$ mechanism mediates effects of $X$ approach'' \citep{lipton2019troubling,hutson2018ai,sculley2018winner,del2019bio}.%
%\footnote{%
%  Of course, important exceptions to this generalization exist \citep{TODO}, and are a focus of subsequent discussion.
%}
%The substance of such concerns, of course, is somewhat allayed where ``new approach $X$'' is grounded in \textit{a priori} empirical or mathematical justification.
%However, in the case of evolutionary algorithms, the fact of biological precedent can partially stand in for explanation of how and why a proposed algorithm would have some desirable properties \citep{sorensen2015metaheuristics}.
% Take, for instance, \citep{TODO}.

Moreover, to a skeptical reader, pervasive appeals to bio-inspiration can come across as window dressing on recycled ideas or as an instance of the naturalistic fallacy \citep{wortmann2020does,sorensen2015metaheuristics}.
% foundation in empiricism and narrative argumentation rather than first-principles mathematical/provable algorithmic rigor; is held back from substantive progress
Such perceptions contribute to a more fundamental concern that evolutionary computation lacks a cohesive, rigorous theoretical framework with first-principles grounding \citep{worzel2003genetic}.
(Concerns in this vein, in part, also persist from more general and longstanding contention between ``scruffy'' and ``neat'' philosophies \citep[p.~16]{jones2008artificial, minsky1991logical}.)
The \textit{ad hoc} nature of evolutionary computation has also been implicated in hindering adoption, on account of numerous configurable and tunable elements confronting new users \citep{oneil2010open}.

One possible response to these concerns is to favor de-emphasizing the evolutionary metaphor, with the goal of deepening and diversifying first principles footing of evolutionary computation \citep{moore2023evolution}.
While we agree that evolutionary computation needs sounder theory, we propose the opposite --- that this challenge can best be addressed by leaning \textit{deeper} into the evolutionary metaphor.
Historically, evolutionary computation researchers have primarily drawn on bio-inspiration to improve the performance of runtime algorithms \citep{banzhaf2006artificial,kumar2003biologically,mcphee2009developmental}.
%  through approaches echoing processes and structures from the natural world
% focuses on emulating of the mechanisms/structure of biology itself that fold into our runtime algorithms.
This approach has been effective, but we argue that it can be taken a step further by looking to the science arising around biological systems, rather than just the biological systems themselves, for inspiration --- to gain methods and theory that characterize and explain EC algorithms, not just improve their performance.
Because evolutionary computation operates on the same principles as biological evolution, substantial amounts of literature on biological theory can be brought to bear on evolution \textit{in silico} \citep{belew1996computation}.
%Enhancing understanding, rather than performance, through bio-inspiration takes a shift in framing: such work can look to the scientific disciplines \textit{studying} biological systems, rather than the biological systems themselves, for inspiration.
As we will review, substantial progress has already been made in this vein --- however, much untapped value remains to be ``arbitraged'' from biological literature.
Subsequently, we will evaluate additional yet-to-be-explored opportunities that might further springboard understanding of evolutionary algorithms off of work in evolutionary biology.
Discussion, in particular, highlights connections between understanding gained from arbitrage of theory and analysis and --- of ultimate importance --- application-oriented objectives of evolutionary computation.

At the most basic level, evolutionary biology and allied fields of population genetics, developmental biology, and ecology complement the focus of evolutionary computation in seeking to explain, rather than optimize (though not always \citep{cobb2013directed,carroll2014applying}).
These life sciences fields have seen radical advances in recent years due to technological leaps in data acquisition and analysis capability \citep{math2018omics,deshpande2024evolution}.
% In particular, research has driven towards model-driven approaches capable of quantitative (as opposed to qualitative) predictions \citep{TODO}.
Also notable is increasing traction gained by the subfield of experimental evolution within evolutionary biology.
Experimental evolution approaches, which observe evolution under controlled conditions, allow detailed inquiry leveraging detailed data collection, replay capabilities, and systematic experimental manipulations \citep{kawecki2012experimental}.

One concern in transposing knowledge from biology to evolutionary computation is the extent to which specificity of theory and methods to biological life might make them poorly applicable to evolutionary computation.
While this is a valid concern, owing to existing needs in biology to stretch abstractions across vast and diverse domains of biological life, we have found there to be a good amount of work that is useful to evolutionary computation.
Indeed, computational artificial life approaches, a close cousin of evolutionary computation, are a popular technique for exploring the generalizability of biological concepts beyond life-as-we-know-it \citep{cleland2013general,langton1989artificial,pennock2007models}. % ELD: Lots more astrobiology citations we could add here
Insofar as content applicable to evolutionary computation exists within biological literature, however, applying that knowledge requires sufficient domain knowledge (1) to identify it and (2) to navigate subtleties in aligning appropriate correspondences with EC.
Given the vastness, context-dependence, and --- at times --- polyonymy of biological theory, interdisciplinary collaborations can be highly productive \citep{goodman2020evolution}.

% such knowledge arbitrage
With investment of effort, engaging theory from evolutionary biology can yield substantial value.
Indeed, a good amount of existing research has already taken such an approach (Table \ref{tab:arbitrage-examples}).
Our review highlights examples across three themes,
\begin{itemize}
  \item genotype-phenotype maps and fitness landscapes (Section \ref{sec:fitness-landscape});
  \item ecology assembly and coexistence theory (Section \ref{sec:ecology}); and
  \item phylogeny for prediction/analysis (Section \ref{sec:phylogeny-analysis}).
\end{itemize}
We also review practical work geared at overlapping evolutionary computation with bioinformatics infrastructure,
\begin{itemize}
  \item sampling-based approaches, which are friendly for decentralized infrastructure (Section \ref{sec:best-effort}); and
  \item interoperation with bioinformatics infrastructure (Section \ref{sec:interoperation}).
\end{itemize}
Additional opportunities for theory arbitrage remain entirely unexplored, however --- which we also outline (Section \ref{sec:opportunities}).

\begin{table}
\footnotesize

\centering
\caption{Representative examples of evolutionary computation work that leverages theory from biology.}
\label{tab:arbitrage-examples}
\renewcommand{\arraystretch}{1.6} % vertical padding between table rows
\begin{tabular}{@{}p{3.2cm}p{5.8cm}p{3.5cm}@{}}
\toprule
\textbf{Theme} & \textbf{Topic} & \textbf{Reference} \\
\midrule
\multirow{2}{\linewidth}{Genotype-Phenotype Maps} & Quantifying Deception: A Case Study in the Evolution of Antimicrobial Resistance & \citep{eppstein2016quantifying} \\
 & TBD & TBD \\
\midrule
\multirow{2}{\linewidth}{Ecology} & Reachability Analysis for Lexicase Selection via Community Assembly Graphs & \citep{dolson2024reachability} \\
 & Ecological theory provides insights about evolutionary computation & \citep{dolson2018ecological} \\
\midrule
\multirow{3}{\linewidth}{Phylogeny Analysis} & What can phylogenetic metrics tell us about useful diversity in evolutionary algorithms? & \citep{hernandez2022can} \\
 & Untangling phylogenetic diversity's role in evolutionary computation using a suite of diagnostic fitness landscapes & \citep{shahbandegan2022untangling} \\
 & Interactions between learning and evolution & \citep{ackley1991interactions} \\
\midrule
\multirow{2}{\linewidth}{Sampling and Tracking Partial Observability} & Methods for Rich Phylogenetic Inference Over Distributed Sexual Populations & \citep{moreno2024methods} \\
 & A Guide to Tracking Phylogenies in Parallel and Distributed Agent-based Evolution Models & \citep{moreno2024guide} \\
\midrule
\multirow{2}{\linewidth}{Interoperation with Bioinformatics Infrastructure} & Data Standards for Artificial Life Software & \citep{lalejini2019data} \\
& alifedata-phyloinformatics-convert & \citep{moreno2024apc} \\
\bottomrule
\end{tabular}
\end{table}


% ELD: Would be good to include some examples here that aren't things we/friends did.

Despite its promise, borrowing from biology should not be taken as a silver bullet, as it is subject to fundamental limitations.
Namely, substantial blind spots remain in the explanatory and predictive power of analyses and theory in ecology and evolution \citep{houlahan2016priority,catford2022addressing,yates2018outstanding} --- although this is best considered as the exception, rather than the rule \citep{lynch2025complexity}.
Not unlike evolutionary computation, deficiency in theory has habituated perennial consternation in evolutionary biology \citep{welch2017wrong}.
Among other factors, blame includes overabundance of identifiable causal factors and the existence of exceptions or complicating factors to nearly every generalization.
Failures to recognize and build on existing work have also been broadly highlighted \citep{lynch2025complexity,beer2024alife}.

These limitations notwithstanding, biological equivalencies seem likely for the preponderance of what explanatory power is possible within EC.
So long as evolutionary algorithms comprise populations of discrete individuals with heritable traits, they will, in some literal sense, instantiate evolutionary processes \citep{pennock2007models}.
Therefore, any systematic theory generalizable across evolutionary computation would likely also hold substantial explanatory power for aspects of biology and \textit{vice versa}.
Notable exceptions, though, arise in unique capabilities within digital evolution to entirely strip out mechanistic elements (e.g., environmental heterogeneity, indirect genotype-phenotype map, etc.) and to achieve perfect, direct observability.
As such, theory established in-house for EC could plausibly lead biology in certain areas, a point we return to in our concluding remarks.
Although EC will doubtlessly continue in contributing new ideas and perspectives on evolution, reciprocal exchange should be expected --- as it seems unlikely for EC, as a smaller field, to profoundly outpace biologists in achieving truly general or cohesive theory.
% ELD: I'm not sold on this last sentence. They've got superior numbers right now, but I actually think evolutionary computation is a really underutilized approach in evolutionary theory

% has profoundly influenced the trajectory of evolutionary algorithm research.
With expanded perspective, the evolution metaphor in EC has value to offer not only in devising new approaches but also in understanding how and why they work.
Looking past biology to harness the science that has arisen around it should continue to be prioritized in advancing the rigor and transparency of evolutionary computation.
By highlighting notable existing steps in this direction, we hope to catalyze continued progress toward these goals.
Calling attention to several yet-untapped correspondences with theory in ecology and evolutionary biology, we hope, also contributes towards this end.
