% ELD: I see what you mean about the framing being tricky. Currently the flow of the intro seems to be:
% - The framing of EC has historically been very biosinspired
% - This is good because of common vocab and charisma
% - But maybe its bad because it leads us to implement algorithms purely on the basis of the naturalistic fallacy
% - This is bad because it hurts EC's credibility
% - We could fix that problem by abandoning bioinspiration, but we're going to aruge that its actually better to lean further into bioinspiration and use it to improve the theoretical backing of EC
% - Further set-up and elaboration on this theory
% 
% I think on first reading it currently leans a little further than ideal into sounding like we're going to argue that the bio-inspiration is bad. Which might alienate the readers we are most interested in attracting (then again, maybe it would attract the readers we most want to change the minds of). It would probably help to be clearer about what we think and what others think. Going to try to make some edits to this effect.

\section{Introduction} \label{sec:introduction}

The metaphor of biological evolution so entirely pervades evolutionary computation (genetic programming, genetic algorithms, etc.) that it can be difficult to discern where bio-inspiration ends and algorithm engineering begins.
Most key innovations in the field were either initially conceived within a bio-inspired framing or were promptly coopted to one.
% ELD: Next sentence could be a little clearer. Exceptions to what? What does it mean that these are core bio-inspired aspects of evolutionary computation?
% Actually, now that I've read the rest of the intro, I wonder if this is too early to go into this much detail.
Although exceptions exist \citep{TODO}, core bio-inspired aspects of evolutionary computation have come to include
\begin{itemize}
  % ELD: My preprint that I still  need to actually publish and/or my 2018 GPTP paper are the obvious citation choices here
  \item diversity maintenance as ecology/negative frequency-dependent selection \citep{TODO},
  % ELD: Citation ideas: Stanley maze paper, Harper SCALP paper, perhaps Anya symbiosis review? Except Anya's, these are all examples rather than review/perspectives, so might need to cite multiple
  \item reciprocal selection as co-evolution \citep{TODO}, and
  % ELD: Your learned genotype-phenotype map stuff would fit well here (I assume you were already planning on that)
  \item search space transforms as genotype-phenotype maps \citep{TODO}.
\end{itemize}
Some have argued that bio-inspiration within evolutionary computing (EC) is valuable simply as a source of impactful, innovative ideas and convenient, intuitive, self-consistent vocabulary \citep{sorensen2015metaheuristics,banzhaf2006artificial}.

Another draw of bio-inspiration is the charisma it lends evolutionary computation \citep{lehman2020surprising}.
The complexity and emergent intelligence found throughout the natural world provides a stunning display of evolution’s profound creative power, and the prospect of capturing any fraction of that power and creativity is fascinating.
% Could also mention that a lot of people find evolution more intuitive 
%The peculiar ubiquity of bio-inspiration in evolutionary computation compared to other ML/AI domains, however, 
Some have questioned, however, whether charisma of the evolutionary approach might also be serving as a preoccupying distraction \citep{moore2023evolution,sorensen2015metaheuristics}.
In particular, given the broader tendency for machine learning researchers to adhere to their preferred models \citep{domingos2012few}, there is a risk that, rather than a means to an end, evolutionary approaches have, to an extent, become an end in and of themselves \citep{woodward2016gp,yampolskiy2018we}.

% ELD: I like this, but I think it's breaking the flow. Maybe it can go somewhere else
%Bio-inspiration, additionally, can be perceived to play into criticism of the \textit{ad hoc} structure of evolutionary computation.
%As typical in outcome-driven AI/ML fields, substantial portions of the EC literature pursue descriptive rather than explanatory objectives --- i.e., ``we found that $X$ approach benefited solution quality'' as opposed to ``we evaluated how $Y$ mechanism mediates effects of $X$ approach'' \citep{lipton2019troubling,hutson2018ai,sculley2018winner,del2019bio}.%
%\footnote{%
%  Of course, important exceptions to this generalization exist \citep{TODO}, and are a focus of subsequent discussion.
%}
%The substance of such concerns, of course, is somewhat allayed where ``new approach $X$'' is grounded in \textit{a priori} empirical or mathematical justification.
%However, in the case of evolutionary algorithms, the fact of biological precedent can partially stand in for explanation of how and why a proposed algorithm would have some desirable properties \citep{sorensen2015metaheuristics}.
% Take, for instance, \citep{TODO}.

Moreover, to a skeptical reader, pervasive appeals to bio-inspiration can come across as window dressing on recycled ideas or as an instance of the naturalistic fallacy \citep{wortmann2020does,sorensen2015metaheuristics}.
% foundation in empiricism and narrative argumentation rather than first-principles mathematical/provable algorithmic rigor; is held back from substantive progress
Such perceptions contribute to a more fundamental concern that evolutionary computation lacks a cohesive, rigorous theoretical framework with first-principles grounding \citep{worzel2003genetic}.
(Concerns in this vein, in part, also persist from more general and longstanding contention between ``scruffy'' and ``neat'' philosophies \citep[p. 16]{jones2008artificial} \citep{minsky1991logical}.)
The \textit{ad hoc} nature of evolutionary computation has also been implicated in hampering its adoption, on account of the numerous arbitrary-tunable elements that confront new users \citep{oneil2010open}.

In response to these concerns, some have called for de-emphasizing the evolutionary metaphor as a path to deepen and diversify first principles footing of evolutionary computation \citep{moore2023evolution}.
% While we agree that evolutionary computation needs sounder theory, we propose that the better solution to the problem would be to do the exact opposite.
% We agree that applicability of the evolution metaphor is a key factor in driving the ills of theory rigor in evolutionary algorithms.
While we agree that evolutionary computation needs sounder theory, we propose instead leaning \textit{deeper} into the evolutionary metaphor as a promising path toward strengthening the foundations of population-based search methods.
Historically, evolutionary computation researchers have primarily drawn on bio-inspiration to improve the performance of runtime algorithms \citep{banzhaf2006artificial,kumar2003biologically,mcphee2009developmental}.
%  through approaches echoing processes and structures from the natural world
% focuses on emulating of the mechanisms/structure of biology itself that fold into our runtime algorithms .
This approach has been effective, but we argue that it can be taken a step further by looking to the science arisen around biological systems, rather than just the biological systems themselves, for inspiration --- to gain methods and theory that characterize and explain EC algorithms, not just improve their performance.
Because evolutionary computation operates on the same principles as biological evolution, large swaths of literature on biological theory could be brought to bear on evolutionary computation.
%Enhancing understanding, rather than performance, through bio-inspiration takes a shift in framing: such work can look to the scientific disciplines \textit{studying} biological systems, rather than the biological systems themselves, for inspiration.
As we will review, substantial progress has already been made in this vein --- however, much more value remains to be ``arbitraged'' from biological literature.
Subsequently, we will evaluate additional yet-to-be-explored opportunities that might further springboard understanding of evolutionary algorithms off of work in evolutionary biology.
Discussion, in particular, highlights connections between understanding gained from arbitraged theory and analysis and --- of ultimate importance --- application-oriented objectives of evolutionary computation.

At the most basic level, evolutionary biology and allied fields of population genetics, developmental biology, and ecology complement the focus of evolutionary computation in seeking to explain, rather than optimize (though not always \citep{cobb2013directed,carroll2014applying}).
These fields have seen radical advances in recent years due to technology-driven growth in data acquisition and analysis capability in the life sciences \citep{TODO}.
In particular, research has driven towards model-driven approaches capable of quantitative (as opposed to qualitative) predictions \citep{TODO}.
Also notable is continuing traction gained by the subfield of experimental evolution within evolutionary biology.
Experimental evolution approaches, which observe evolution under controlled conditions, allow detailed inquiry that avails direct observability, replay capabilities, and systematic experimental manipulations \citep{kawecki2012experimental}.

One concern in transposing from biology to evolutionary computation is whether theory and methods are specific enough to biological life so as to be poorly applicable or informative to evolutionary computation.
While it is true that certain particulars are of limited general applicability, owing to practical needs to incorporate abstractions for tractability and generalizability across domains of biological life, we have found there to be a good amount of work that is useful to evolutionary computation.
Indeed, computational artificial life approaches, a close cousin of evolutionary computation, are a popular technique for exploring the generalizability of biological findings beyond life-as-we-know-it is \citep{cleland2013general}. % ELD: Lots more astrobiology citations we could add here
Insofar as value applicable to evolutionary computation exists within biological literature, however, applying that knowledge requires sufficient domain knowledge (1) to identify it and (2) to navigate subtleties in aligning appropriate correspondences with EC.
Given the vastness, density, and --- at times --- polyonymy of biological theory, interdisciplinary collaborations can be highly productive \citep{goodman2020evolution}.

% such knowledge arbitrage
With investment of effort, engaging theory and methods from evolutionary biology can yield substantial value.
Indeed, a good amount of existing research has already taken such an approach.
Our review covers
\begin{itemize}
  \item genotype-phenotype maps \citep{TODO};
  \item ecology assembly theory \citep{dolson2024reachability}; %ELD: Could maybe expand this to coexistence theory too
  \item phylogeny for prediction/analysis \citep{hernandez2022can,shahbandegan2022untangling,moreno2024ecology};
  \item alife data standards: interoperation with bioinformatics infrastructure \citep{lalejini2019data,moreno2024apc}; and
  \item reconstruction-based tracking \citep{moreno2022hstrat,moreno2024guide}.
\end{itemize}
Additional opportunities for theory arbitrage remain entirely unexplored, however --- which we also outline.

% ELD: Would be good to include some examples here that aren't things we/friends did. Some ideas:
% - Brandon Ogbunu/Maggie Epstien's work on quantifying deceptiveness and predicting evolutionary trajectories
% - I'm confident Stephanie Forrest has done something worth including here although I'd need to go read (she actually might have written a review/perspective that we should be citing)
% - There's got to be more. I'll keep thinking

Despite its promise, borrowing from biology should not be taken as a silver bullet.
There are fundamental limitations in cohesiveness and universality attainable through such an approach.
% ELD: Home turf is probably a little too informal
Namely, even taken alone, substantial blind spots remain in the explanatory and predictive power analyses and theory in ecology and evolution \citep{TODO}.
% ELD: Maybe point out that they've still got multiple orders of magnitude more theory than EC? Evolutionary biology could do better, but even within biology evolution is often seen as one of the fields that has been more successful at embracing theory (ecological theorists get envious)
Not unlike evolutionary computation, deficiency in theory has habituated as a source of perennial consternation in evolutionary biology \citep{welch2017wrong}.
Among other factors, blame includes overabundance of identifiable contributing causal factors and the consistency of exceptions or complicating factors to nearly every generalization.
% ELD: Not sure how to put this, but I feel like that's true for one type of rule, but there's another type of rule where its not. E.g. there are no exceptions to the fundamental principle of adaptation by natural selection, even if there are circumstances where it gets more complicated

% ELD: Next sentence is a little clunky
Despite limitations of biological analysis and theory, arbitrage from biology seems likely to yield a substantial fraction what explanatory power is ultimately possible to establish within EC.
So long as evolutionary algorithms comprise populations of discrete individuals with heritable traits, they will, in some literal sense, instantiate evolutionary processes \citep{pennock2007models}.
Therefore, any systematic theory generalizable across evolutionary computation would likely also hold substantial explanatory power for aspects of biology and \textit{vice versa}.
Notable exceptions, though, arise in EC's unique capabilities to entirely strip out elements e.g., of environmental heterogeneity or genotype-phenotype map and to achieve perfect, direct observability unique to digital evolution.
As such, theory established in-house for EC could plausibly lead biology in certain areas, a point we return to in our concluding remarks.
Although EC will doubtlessly continue in contributing new ideas and perspectives on evolution, it seems unlikely, in most areas, for evolutionary computation to profoundly outpace biology in achieving truly general or cohesive theory.
% ELD: I'm not sold on this last sentence. They've got superior numbers right now, but I actually think evolutionary computation is a really undertapped approach in evolutionary theory

% has profoundly influenced the trajectory of evolutionary algorithm research.
With expanded perspective, the evolution metaphor in EC has value to offer not only in devising new approaches, but also in understanding how and why they work.
Looking past biology to harness the science that has arisen around it should continue to be pursued as a priority in advancing the rigor and transparency of evolutionary computation.
By highlighting notable existing steps in this direction, we hope to catalyze continued progress toward these goals.
Calling attention to several yet-untapped correspondences with theory in ecology and evolutionary biology, we hope, also contributes towards this end.
