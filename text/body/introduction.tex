\section{Introduction} \label{sec:introduction}

It can be difficult to discern where bio-inspiration of genetic programming and genetic algorithms ends and algorithm engineering begins.
Narrative and vocabulary of biological evolution pervades evolutionary computation.
Key innovations in the field were conceived, or were promptly co-opted, within a bio-inspired framing.
Exceptions exist \citep{TODO}, but take for instance,
\begin{itemize}
  \item diversity maintenance as ecology/negative frequency-dependent selection \citep{TODO},
  \item reciporical selection effects as co-evolution \citep{TODO}, and
  \item search space transforms as genotype-phenotype maps \citep{TODO}.
\end{itemize}
Under this framing, prevalence of bio-inspiration within evolutionary computing simply attributes to value as a source of impactful, innovative ideas and convenient, intuitive, self-consistent vocabulary \citep{sorensen2015metaheuristics,banzhaf2006artificial}.

% Under this framing, pervasion of bio-inspiration could be considered to have stuck simply owing to
% Clearly, bio-inspiration has been productive in provided

Another draw of bio-inspiration is the enthusiastic charisma it lends evolutionary computation \citep{lehman2020surprising}.
Oportunity to harness even a fraction of evolution's creative power, much less to see processes responsible for so much of the world around us play out in real time, serves as a major draw to the field.
The peculiar ubiquity of bio-inspiration in evolutionary computation compared to other ML/AI domains, however, brings into question the extent to which charisma of the evolutionary approach might also be serving as a preoccupying distraction \citep{moore2023evolution,sorensen2015metaheuristics}.
In particular, given the broader tendency for machine learning researchers to adhere to their preferred models \citep{domingos2012few}, it is reasonable to argue that, rather than a means to an end, evolutionary approaches have to an extent become an end in and of themselves \citep{woodward2016gp,yampolskiy2018we}.

Bio-inspiration, additionally, can be perceived to play into criticism of the \textit{ad hoc} structure of the field.
As typical in outcome-driven AI/ML fields, substantial portions of the literature pursue descriptive rather than explanatory objectives --- i.e., ``we found that $X$ approach benefited solution quality'' as opposed to ``we evaluated how $Y$ mechanism mediates effects of $X$ approach on solution quality'' \citep{lipton2019troubling,hutson2018ai,sculley2018winner,del2019bio}.%
\footnote{%
  Of course, important exceptions to this generalization exist, and are a focus of subsequent discussion.
}
The substance of such concerns, of course, is somewhat allayed where ``new approach $X$'' is grounded in some empirical or mathematical justification.
However, in the case of evolutionary algorithms, the fact of biological precedent can partially stand in for explanation of how and why a proposed algorithm would have some desirable properties \citep{sorenson2015metaheuristics}.
% Take, for instance, \citep{TODO}.

To a skeptical reader, pervasive appeals to bio-inspiration lend a tint of hand-waving, naturalistic fallacy, and superficial churn \citep{wortmann2020does,sorensen2015metaheuristics}.
% foundation in empiricism and narrative argumentation rather than first-principles mathematical/provable algoritmic rigor; is held back from substinative progress
Such perceptions contribute to a more fundamental criticism that evolutionary computation lacks of cohesive, rigorous theoretical framework with first-principles grounding \citep{TODO}.
(Concerns in this vein, in part, also persist from more general and longstanding contention between ``scruffy'' and ``neat'' philosophies \citep[p. 16]{jones2008artificial} \citep{minsky1991logical}.)
The \textit{ad hoc} nature of evolutionary computation has also been identied in diminishing its outside adoption, on account of the numerous of arbitrary-tunable elements that confront new users \citep{oneil2010open}.

One path forward, therefore, calls for exploration of approaches de-emphasizing the evolutionary metaphor in order to deepen and diversify first principles footing \citep{moore2023evolution}.
% We agree that applicability of the evolution metaphor is a key factor in driving the ills of theory rigor in evolutionary algorithms.
However, here, we highlight leaning \textit{deeper} into the evolutionary metaphor as a complementary path forward in strengthening the foundations of evolutionary computation.
Typical, and highly productive, practice has drawn on bio-inspiration to improve runtime algorithms through approaches echoing processes and structures from the natural world \citep{banzhaf2006artificial,kumar2003biologically}.
% focuses on emulating of the mechanisms/structure of biology itself that fold into our runtime algorithms .
Enhancing understanding, rather than performance, requires an alternate framing: such work looks to the scientific disciplines \textit{studying} biological systems, rather than the biological systems themselves, for inspiration.
As we will review, substantial progress has already been made in this vein --- however, much more value remains to be ``arbitraged.''
In closing, we will suggest opportunities areas to further springboard understanding of evolutionary algorithms off of work in evolutionary biology, as well.
We will also discuss connections to be made between understanding gained from this ``third way'' and, of ultimate importance, the application-oriented objectives of evolutionary computation.

% has profoundly influenced the trajectory of evolutionary algorithm research.

By their very nature, evolutionary biology is complementary to evolutionary computation as a field with typically (but not always CITE directed evolution) descriptive/analytical field while genetic programming is more engineering driven;
Contemporary biology provides vast, rich opportunity for arbitrage.
To name a few pertinent subdisciplines population genetics, evolutionary biology, developmental biology, and ecology
It has matured into a highly quantitative, model-driven field.
The advent of experimental evolution has also been opening up inquiry that is driven by falsifiable hypotheses and experimental manipulations.
while it is true that significant aspects of methodology are speciric to the particulars of biology, with creative and substantial tweaks some surprising aspects of biological study can become the genetalizabile.
For instance, software for phylogenetic reconstruction is tailored to the four letter nucleotide alphabet and reconstruction is a problem specific to biology itself because in simulation we in principle have access to perfect information.
However, recent work has engineered analogs of reconstruction approwcjrs to achieve highly scalable phylogenetic tracking.
And the rest pf the bioinformstics infrastructure for measuring and intetpreting phylogenetic structure translates very cleanly.

Taking advantage of biological theory not only requires dealing with subtle issues alogning reasonable correspondences, but also because there is so much so vast it also requires developing familiarity with often dense and technical topics outside of typical comfort/familiarity zone.
in addition to reading up, investing in making contacts and collaborations is another important resource.
the BEACON center for the study of evolution in action is an exemplar in this regard \citep{goodman2020evolution}.

if we invest the effort to become familiar with them, there is a lot of value that can be arbitraged from biologal literature.
indeed this had already been a rich vein of research in evolutionary algorithms.
in this review, we cover
\begin{itemize}
  \item genotype-phenotype maps
  \item ecology assembly theory \citep{dolson2024reachability}
  \item phylogeny for prediction/analysis \citep{hernandez2022can,shahbandegan2022untangling}
  \item alife data standards: interoperation with bioinformatics infrastructure \citep{lalejini2019data,moreno2024apc}
  \item reconstruction-based tracking \citep{moreno2022hstrat,moreno2024ecology}
\end{itemize}
beyond the work surveued here there is substantial untapped potential value in these areas wnd we additionwlly suggest areas where arbitrage from biology could be profitable in the future

The limitations of this approach should be noted --- it will only have a limited effect in developing a cohesive theory for genetic programming and genetic algorithms.
this is due to there being limited success in developing cohesive theory for evolutionary biology.
Like for evolutionary algoritmns, deficiency in theory has been a source of perennial consternation \citep{welch2017wrong}.
it has been argued that this is due to intrsic challenges in the subject matter LIKE X Y Z
This WORD approach is likely as fruitful as any other.
Because so long as evolutionary algorithms bear the fundamental structure of (borrow from Emily), they will be instances of evolution \citep{pennock2007models}.
Therefore, by a sort of computational reproducibility argument, a systemstic theory of genetic algorithms/genetic programming would have substantial explanatory power for aspects of biology.
Given the scope and stature of the biological science enterprise, it seems unlikely for any transformative unifying theory to result.
However, we do argue that there is great potential for theory arbitrage to be bidirectional, and we identify some areas where evolutionary algoritms could make important contributipns to the corpus of biological literature.

Ultimately, we argue that with new perspective the evolution metaphor in genetic programming and genetic algorithms has yet more to give.
For genetic programming and genetic algorithm methods to advance to the next level of rigor and effectiveness, we should look beyond the biological processes themselves and also take inspiration from the fields studying it.
By highlighting notable classic and more recent activity in this direction, we hope to draw attention to potential in the direction and catalyze progress.
This is why we close by discussing areas we see as priorities to exploit for arbitrage.
