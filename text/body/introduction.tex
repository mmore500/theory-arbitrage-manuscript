\section{Introduction} \label{sec:introduction}

Trappings of biological evolution so entirely pervade genetic programming and genetic algorithms that it can be difficult to discern where bio-inspiration ends and algorithm engineering begins.
Key innovations in the field were initially conceived --- or were promptly co-opted --- within a bio-inspired framing.
Although exceptions exist \citep{TODO}, core bio-inspired aspects of evolutionary computation have come to include
\begin{itemize}
  \item diversity maintenance as ecology/negative frequency-dependent selection \citep{TODO},
  \item reciprocal selection as co-evolution \citep{TODO}, and
  \item search space transforms as genotype-phenotype maps \citep{TODO}.
\end{itemize}
In one possible framing, the prevalence of bio-inspiration within evolutionary computing (EC) simply attributes to value as a source of impactful, innovative ideas and convenient, intuitive, self-consistent vocabulary \citep{sorensen2015metaheuristics,banzhaf2006artificial}.

Another draw of bio-inspiration, however, is the enthusiastic charisma it lends evolutionary computation \citep{lehman2020surprising}.
a major draw to the field is the notion to harness even a fraction of evolution's creative power, much less to see processes responsible for so much of the world around us play out in real time.
The peculiar ubiquity of bio-inspiration in evolutionary computation compared to other ML/AI domains, however, brings into question the extent to which charisma of the evolutionary approach might also be serving as a preoccupying distraction \citep{moore2023evolution,sorensen2015metaheuristics}.
In particular, given the broader tendency for machine learning researchers to adhere to their preferred models \citep{domingos2012few}, it is reasonable to argue that, rather than a means to an end, evolutionary approaches have, to an extent, become an end in and of themselves \citep{woodward2016gp,yampolskiy2018we}.

Bio-inspiration, additionally, can be perceived to play into criticism of the \textit{ad hoc} structure of evolutionary computation.
As typical in outcome-driven AI/ML fields, substantial portions of the EC literature pursue descriptive rather than explanatory objectives --- i.e., ``we found that $X$ approach benefited solution quality'' as opposed to ``we evaluated how $Y$ mechanism mediates effects of $X$ approach'' \citep{lipton2019troubling,hutson2018ai,sculley2018winner,del2019bio}.%
\footnote{%
  Of course, important exceptions to this generalization exist \citep{TODO}, and are a focus of subsequent discussion.
}
The substance of such concerns, of course, is somewhat allayed where ``new approach $X$'' is grounded in \textit{a priori} empirical or mathematical justification.
However, in the case of evolutionary algorithms, the fact of biological precedent can partially stand in for explanation of how and why a proposed algorithm would have some desirable properties \citep{sorenson2015metaheuristics}.
% Take, for instance, \citep{TODO}.

To a skeptical reader, pervasive appeals to bio-inspiration lend a tint of hand-waving, naturalistic fallacy, and superficial churn \citep{wortmann2020does,sorensen2015metaheuristics}.
% foundation in empiricism and narrative argumentation rather than first-principles mathematical/provable algorithmic rigor; is held back from substantive progress
Such perceptions contribute to a more fundamental criticism that evolutionary computation lacks a cohesive, rigorous theoretical framework with first-principles grounding \citep{TODO}.
(Concerns in this vein, in part, also persist from more general and longstanding contention between ``scruffy'' and ``neat'' philosophies \citep[p. 16]{jones2008artificial} \citep{minsky1991logical}.)
The \textit{ad hoc} nature of evolutionary computation has also been implicated in hampering adoption, on account of the numerous arbitrary-tunable elements that confront new users \citep{oneil2010open}.

One path forward calls for consideration of approaches de-emphasizing the evolutionary metaphor, in order to deepen and diversify first principles footing \citep{moore2023evolution}.
% We agree that applicability of the evolution metaphor is a key factor in driving the ills of theory rigor in evolutionary algorithms.
In contrast, here, we highlight how leaning \textit{deeper} into the evolutionary metaphor provides a complementary path toward strengthening the foundations of population-based search methods.
Typical, and highly productive, practice in evolutionary computation has drawn on bio-inspiration to augment runtime algorithms \citep{banzhaf2006artificial,kumar2003biologically}.
%  through approaches echoing processes and structures from the natural world
% focuses on emulating of the mechanisms/structure of biology itself that fold into our runtime algorithms .
Enhancing understanding, rather than performance, through bio-inspiration takes a shift in framing: such work can look to the scientific disciplines \textit{studying} biological systems, rather than the biological systems themselves, for inspiration.
As we will review, substantial progress has already been made in this vein --- however, much more value remains to be ``arbitraged'' from biological literature.
Subsequently, we will evaluate additional yet-to-be-explored opportunities that might further springboard understanding of evolutionary algorithms off of work in evolutionary biology.
Discussion, in particular, highlights connections between understanding gained from arbitraged theory and analysis and --- of ultimate importance --- application-oriented objectives of evolutionary computation.

At the most basic level, evolutionary biology and allied fields of population genetics, developmental biology, and ecology complement the focus of evolutionary computation in seeking to explain, rather than optimize (though not always \citep{cobb2013directed,carroll2014applying}).
These fields have seen radical advances in recent years due to technology-driven growth in data acquisition and analysis capability in the life sciences \citep{TODO}.
In particular, research has driven towards model-driven approaches capable of quantitative (as opposed to qualitative) predictions \citep{TODO}.
Also notable is continuing traction gained by the subfield of experimental evolution within evolutionary biology.
Experimental evolution approaches, which observe evolution under controlled conditions, allow detailed inquiry that avails direct observability, replay capabilities, and systematic experimental manipulations \citep{kawecki2012experimental}.

One concern in transposing from biology to evolutionary computation is the extent to which theory and methods are specific enough to biological life so as to be poorly applicable or informative to evolutionary computation.
While it is the case that certain particulars are of limited general applicability, owing to practical needs to incorporate abstractions for tractability and generalizability across domains of biological life, we have found there to be a good amount of work that is useful to evolutionary computation.
Indeed, the question of generalizability beyond life-as-we-know-it is, of itself, a topic of substantial attention --- with computational artificial life approaches tapped directly in research \citep{cleland2013general}.
Insofar as value applicable to evolutionary computation exists within biological literature, however, applying that knowledge requires sufficient domain knowledge (1) to identify it and (2) to navigate subtleties in aligning appropriate correspondences with EC.
Given the vastness, density, and --- at times --- polyonymy of biological theory, interdisciplinary collaborations can be highly productive \citep{goodman2020evolution}.

% such knowledge arbitrage
With investment of effort, engaging theory and methods from evolutionary biology can yield substantial value.
Indeed, a good amount of existing research has already taken such an approach.
Our review covers
\begin{itemize}
  \item genotype-phenotype maps \citep{TODO};
  \item ecology assembly theory \citep{dolson2024reachability};
  \item phylogeny for prediction/analysis \citep{hernandez2022can,shahbandegan2022untangling};
  \item alife data standards: interoperation with bioinformatics infrastructure \citep{lalejini2019data,moreno2024apc}; and
  \item reconstruction-based tracking \citep{moreno2022hstrat,moreno2024ecology}.
\end{itemize}
Additional opportunities for theory arbitrage remain entirely unexplored, however --- which we also outline.

Despite its promise, borrowing from biology should not be taken as a silver bullet.
There are fundamental limitations in cohesiveness and universality attainable through such an approach.
Namely, even on home turf, substantial blind spots remain in the explanatory and predictive power analyses and theory in ecology and evolution \citep{TODO}.
Not unlike evolutionary computation, deficiency in theory has habituated as a source of perennial consternation in evolutionary biology \citep{welch2017wrong}.
Among other factors, blame includes numerosity of identifiable contributing causal factors and the consistency of exceptions to nearly every rule.

Despite limitations of biological analysis and theory, an arbitrage approach seems likely to capture a substantial portion of what explanatory power is ultimately possible to establish within EC.
So long as evolutionary algorithms comprise populations of discrete individuals with heritable traits, they will, in some literal sense, instantiate evolutionary processes \citep{pennock2007models}.
Therefore, any systematic theory generalizable across evolutionary computation would likely also hold substantial explanatory power for aspects of biology.
Notable exceptions, though, arise in EC's unique capabilities to entirely strip out elements e.g., of environmental heterogeneity or genotype-phenotype map and to achieve perfect, direct observability unique to digital evolution.
As such, theory established in-house for EC could plausibly lead biology in certain areas, a point we return to in our concluding remarks.
However, in most areas, it seems unlikely for evolutionary computation to outpace biology in achieving truly general or cohesive theory.

% has profoundly influenced the trajectory of evolutionary algorithm research.
With expanded perspective, the evolution metaphor in EC has value to offer not only in devising new approaches, but also in understanding how and why they work.
Looking past biology to harness the science that had arisen around it should continue to be pursued as a priority in advancing the rigor and transparency of evolutionary computation.
By highlighting notable existing steps in this direction, we hope to catalyze continued progress toward these goals.
Calling attention to several yet-untapped correspondences with theory in ecology and evolutionary biology, we hope, also contributes towards this end.
