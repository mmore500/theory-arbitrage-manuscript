\section{Introduction} \label{sec:introduction}

It can be difficult to discern where bio-inspiration of genetic programming and genetic algorithms ends and algorithm engineering begins.
Narrative and vocabulary of biological evolution pervades evolutionary computation.
Key innovations in the field were conceived, or were promptly co-opted, within a bio-inspired framing.
Exceptions exist \citep{TODO}, but take for instance,
\begin{itemize}
  \item diversity maintenance as ecology/negative frequency-dependent selection \citep{TODO},
  \item reciporical selection as co-evolution \citep{TODO}, and
  \item search space transforms as genotype-phenotype maps \citep{TODO}.
\end{itemize}
Under this framing, prevalence of bio-inspiration within evolutionary computing simply attributes to value as a source of impactful, innovative ideas and convenient, intuitive, self-consistent vocabulary \citep{sorensen2015metaheuristics,banzhaf2006artificial}.

% Under this framing, pervasion of bio-inspiration could be considered to have stuck simply owing to
% Clearly, bio-inspiration has been productive in provided

Another draw of bio-inspiration is the enthusiastic charisma it lends evolutionary computation \citep{lehman2020surprising}.
Oportunity to harness even a fraction of evolution's creative power, much less to see processes responsible for so much of the world around us play out in real time, serves as a major draw to the field.
The peculiar ubiquity of bio-inspiration in evolutionary computation compared to other ML/AI domains, however, brings into question the extent to which charisma of the evolutionary approach might also be serving as a preoccupying distraction \citep{moore2023evolution,sorensen2015metaheuristics}.
In particular, given the broader tendency for machine learning researchers to adhere to their preferred models \citep{domingos2012few}, it is reasonable to argue that, rather than a means to an end, evolutionary approaches have to an extent become an end in and of themselves \citep{woodward2016gp,yampolskiy2018we}.

Bio-inspiration, additionally, can be perceived to play into criticism of the \textit{ad hoc} structure of the field.
As typical in outcome-driven AI/ML fields, substantial portions of the literature pursue descriptive rather than explanatory objectives --- i.e., ``we found that $X$ approach benefited solution quality'' as opposed to ``we evaluated how $Y$ mechanism mediates effects of $X$ approach on solution quality'' \citep{lipton2019troubling,hutson2018ai,sculley2018winner,del2019bio}.%
\footnote{%
  Of course, important exceptions to this generalization exist, and are a focus of subsequent discussion.
}
The substance of such concerns, of course, is somewhat allayed where ``new approach $X$'' is grounded in some empirical or mathematical justification.
However, in the case of evolutionary algorithms, the fact of biological precedent can partially stand in for explanation of how and why a proposed algorithm would have some desirable properties \citep{sorenson2015metaheuristics}.
% Take, for instance, \citep{TODO}.

To a skeptical reader, pervasive appeals to bio-inspiration lend a tint of hand-waving, naturalistic fallacy, and superficial churn \citep{wortmann2020does,sorensen2015metaheuristics}.
% foundation in empiricism and narrative argumentation rather than first-principles mathematical/provable algoritmic rigor; is held back from substinative progress
Such perceptions contribute to a more fundamental criticism that evolutionary computation lacks of cohesive, rigorous theoretical framework with first-principles grounding \citep{TODO}.
(Concerns in this vein, in part, also persist from more general and longstanding contention between ``scruffy'' and ``neat'' philosophies \citep[p. 16]{jones2008artificial} \citep{minsky1991logical}.)
The \textit{ad hoc} nature of evolutionary computation has also been identied in diminishing its outside adoption, on account of the numerous of arbitrary-tunable elements that confront new users \citep{oneil2010open}.

One path forward, therefore, calls for exploration of approaches de-emphasizing the evolutionary metaphor in order to deepen and diversify first principles footing \citep{moore2023evolution}.
% We agree that applicability of the evolution metaphor is a key factor in driving the ills of theory rigor in evolutionary algorithms.
However, here, we highlight leaning \textit{deeper} into the evolutionary metaphor as a complementary path forward in strengthening the foundations of evolutionary computation.
Typical, and highly productive, practice has drawn on bio-inspiration to improve runtime algorithms through approaches echoing processes and structures from the natural world \citep{banzhaf2006artificial,kumar2003biologically}.
% focuses on emulating of the mechanisms/structure of biology itself that fold into our runtime algorithms .
Enhancing understanding, rather than performance, requires an alternate framing: such work looks to the scientific disciplines \textit{studying} biological systems, rather than the biological systems themselves, for inspiration.
As we will review, substantial progress has already been made in this vein --- however, much more value remains to be ``arbitraged.''
In closing, we will suggest opportunities areas to further springboard understanding of evolutionary algorithms off of work in evolutionary biology, as well.
We will also discuss connections to be made between understanding gained from this ``third way'' and, of ultimate importance, the application-oriented objectives of evolutionary computation.

On the most fundamental level, evolutionary biology and allied fields of population genetics, developmental biology, and ecology complement the focus of evolutionary computation in seeking to explain, rather than optimize (though not always \citep{cobb2013directed,carroll2014applying}).
In recent years, these fields have seen radical advances owing to technology-driven growth in capability for data acquisition and analysis \citep{TODO}.
In particular, there has been a drive towards model-driven approaches capable of quantitative, as opposed to qualitative, predictions \citep{TODO}.
Within evolutionary biology, the subfield of of experimental evolution continiues to gain traction.
These approaches, which study evolution as it occurs under controlled conditions, allows detailed inquiry taking advantage of direct observability, replay capabilities, and systematic experimental manipulations \citep{kawecki2012experimental}.

One concern in transposing from biology to evolutionary computation is the extent to which theory and methods are specific enough to biological life so as to be poorly applicable or informative to evolutionary computation.
While it is the case that certain particulars are of limited general applicability, owing for pratical needs to incorporate abstractions for tractability and generalizability across forms of biological life, we have found there to be a good amount of work that is useful to evolutionary computation.
Indeed, the question of generalizability beyond life-as-we-know it is, of itself, a topic of inquiry, with computational artificial life approaches tapped directly in research \citep{cleland2013general}.
Insofar as value applicable to evolutionary computation appears within existing literature, however, applying that knowledge requires sufficient domain knowledge to identify it and navigate subtleties in aligning appropriate correspondences to the question at hand.
Given the vastness, density, and --- at times --- polyonymy of biological theory, interdisciplinary collaborations can be highly productive \citep{goodman2020evolution}.

With investment of effort to engage theory and methods from evolutionary biology, such knowledge arbitrage can prove highly impactful.
Indeed, a good amount of existing research has already taken such an approach.
In this review, we cover
\begin{itemize}
  \item genotype-phenotype maps \citep{TODO};
  \item ecology assembly theory \citep{dolson2024reachability};
  \item phylogeny for prediction/analysis \citep{hernandez2022can,shahbandegan2022untangling};
  \item alife data standards: interoperation with bioinformatics infrastructure \citep{lalejini2019data,moreno2024apc}; and
  \item reconstruction-based tracking \citep{moreno2022hstrat,moreno2024ecology}.
\end{itemize}
Valuable opportunities for theory arbitrage remain entirely unexplored, however, which we also outline.

Despite its promise, borrowing from biology should not be approached as a silver bullet.
There are fundamental limitations in the potential for this cohesiveness and universality attainable through such an approach.
Namely, even on home turf, substantial blind spots remain in the explanatory and predictive power analyses and theory in ecology and evolution \citep{TODO}.
Not unlike evolutionary computation, deficiency in theory has habituated a source of perennial consternation in evolutionary biology \citep{welch2017wrong}.
Among other factors, blame includes the numerosity of identifiable contributing causal factors and the consistency of exceptions to nearly every rule.

Despite limitations of analysis and theory in biology, an arbitrage approach seems likely to capture a substantial portion of what explanatory power is possible to establish.
So long as evolutionary algorithms comprise populations of discrete individuals with heritable traits, they will in some literal sense instantiate evolutionary processes \citep{pennock2007models}.
Therefore, systematic theory generalizable across evolutionary computation would be expected to hold substantial explanatory power for aspects of biology, as well.
Notable exceptions, though, arise is in capability to entirely strip out elements e.g., of environmental heterogeneity or genotype-phenotype map and to achieve perfect, direct observability unique to digital evolution.
Evolutionary computation is thus positioned to lead in establishing theory in certain areas, a point we return to in our concluding remarks.
However, in most areas, it seems unlikely for evolutionary computation to achieve truly general or cohesive theory where biology hasn't.

% has profoundly influenced the trajectory of evolutionary algorithm research.
With expanded perspective, the evolution metaphor in genetic programming and genetic algorithms has value to offer not only in devising new approaches but in understanding how and why they work.
Looking past biology to harness the science that had arisen around it should continue to be pursued ad a priority in advancing the rigor and transparency of evolutionary computation.
By highlighting notable steps in this direction, we hope to and catalyze continued progress toward these goals.
To this end, we additionally emphasize several yet untapped correspondences with theory in ecology and evolutionary biology ripe for arbitrage.
