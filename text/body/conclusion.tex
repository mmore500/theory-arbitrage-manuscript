% or even be used to optimize biological organisms through directed evolution inspired by evolutionary computaiton approaches \citep{lalejini2022artificial}

\section{Conclusion} \label{sec:conclusion}

The sophosticated functionality of the natural world --- still, in many cases, profoundly outstripping the efficiency and quality of material science, kinetic properties, and control strategies devised through engineering --- strongly evidences the creative power of evolution, and lends it substantial charisma.
The notion of being seduced by this charisma is a reasonable criticism of evolutionary computation \citep{moore2023evolution,woodward2009evolution}.
The evolutionary metaphor has been prodictive, although its difficult to say whether this is because of its objective value or its success at capturing the attention of the field.

We argue that one answer to firming up the theoretical foundation of evolutonary computation is to reach further into the biological metaphor. The thought is to reach past the biology itself and instead engage  the explanatory analytical methods and theoretical constructs surrounding  evolution biology and allied fields.
Although scientific literature is somewhat less charismatic, and can be more obscure to become acquainted with, it can help us with our ultimate objectives of explaining how aspects of our algoritmns intetact, diagnose poor performance, and predict appropriate approaches to suit particular domains.

In this work, we have reviewed several classic and recent examples of reaching into the biological literature to improve our visibility into application-oriented evolutionary computation.
These pieces of work are just first steps into this direction.
Important extensions remain to flesh them out and connect them to real-world use cases.
Further, they are steps in but a small portion of possible directions where theory and analysis from biology could be productively arbitraged to evolutionary computation.
In our closing section, we have highlighted a few possibilities.

Productive exchange between biology and evolutionary computation is not a one way street.
For one, digital organisms have served as a productive model system for evolution experiments.
However, there is untapped potential for arbitrage from more application-oriented genetic programming and genetic algorithms.
One particular area where evolutionary computation could benefit biological science is with respect to teasing apart the relationship between genotype-phentoype map and evolvability.
Owing to their artificial nature, evolutionry computation has confronted the question of evolvability much more directly than biology where it comes baked in ``for free'' \citep{kirschner1998evolvability}.
In particular, it is well-understood that tuning genotype-phenotype map is a critical consideration, and there has been a body of work on understanding what properties of genotype-phenotype maps lend well to adaptive evolution (ADD CITES).
In more recent years, a promising vein of work in evolutionary computation has picked up on how harnessing unsupervised learning (e.g., autoencoders, LLMs) to generate evolvable genotype-phenotype maps \citep{moreno2018understanding,bradley2024openelm,wittenberg2023denoising,MORECITES}.
Correspondingly, in biology, there has been a fledgling realization that evolvability could be interpreted as an unsupervised learning process --- driven forward, in particular, through collaboration with evolutionary computation practicioners \citep{kouvaris2017evolution,szilagyi2020phenotypes}.

%TODO this needs a rewrite
The thoroughness of evolutionary computation's tethering to bio-inspiration is truly remarkable.
% synthesis of deep learning and neuroscience \citep{richards2019deep,marblestone2016toward}
For instance, although deep learning is also rooted in a biological metaphor, there hasn't been as much fruitful impact translating neuroscience mechanisms to mainline research (although there have been efforts on certain frontiers CITE CITE CITE \citep{furber2014spinnaker}).
We have the potential to further leverage the biological metaphor to our advantage.
