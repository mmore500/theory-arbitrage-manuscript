% or even be used to optimize biological organisms through directed evolution inspired by evolutionary computaiton approaches \citep{lalejini2022artificial}

\section{Conclusion} \label{sec:conclusion}

The sophistication of natural organisms --- still, in many cases, thoroughly unrivaled by artificial engineering --- strongly evidences the creative potential of evolutionary computation, and also lend it unique charisma.
% The notion of being seduced by this charisma is a reasonable criticism of evolutionary computation \citep{moore2023evolution,woodward2009evolution}.
The evolutionary metaphor in genetic algorithms and genetic programming has proven highly productive, both in guiding research advances and also, more simply, in capturing research attention.
Despite reasonable criticisms of metaphor-driven algorithm development \citep{moore2023evolution,sorensen2015metaheuristics}, we argue compelling potential exists to strengthen the foundations of evolutionary computation by instead leaning further into the evolution metaphor --- by engaging methods and theory constructs from evolutionary biology and allied fields.

In this review, we have highlighted notable examples where reaching into the biological literature has improved visibility into application-oriented evolutionary computation.
Such research, much of it very recent, reflects only first steps.
Important extensions remain to be fleshed out and connected to \textit{bona fide} real-world use cases.
Further, reviewed work touches only a small fraction of promising directions for arbitrage from biology to evolutionary computation.
In Section \ref{sec:opportunities}, we have highlighted several possibilities.

Productive exchange between biology and evolutionary computation is a two-way street.
Already, digital organisms find use cases in experiments conducted in conjunction with \textit{in vivo} inquiry \citep{sanjun2007selection,wilke2001evolution,hindr2012new}.
However, we believe that application-oriented genetic programming and genetic algorithms have special value to offer in the realm of genotype-phenotype maps and evolvability.
Owing to its artificial nature, genotype-phenotype maps --- much less, maps with strong evolvabilty --- do not come baked into evolutionary computation \textit{a priori} \citep{kirschner1998evolvability}.
As such, EC research has invested significant effort in teasing apart how the properties of a genotype-phenotype map influence outcomes from adaptive evolution \citep{banzhaf1994genotype,hu2010evolvability,whigham2017mapping}.
A particularly promising line of EC work has sprung up in investigating how to harness unsupervised learning (e.g., autoencoders, LLMs) to generate evolvable genotype-phenotype maps \citep{lehman2023evolution,moreno2018learning,bentley2022evolving,gaier2020discovering,wittenberg2023denoising}.
Corresponding work in biology, however, is only fledgling.
Indeed, such work considering evolvability as an unsupervised learning process has notably already been driven forward through collaboration with evolutionary computation practitioners \citep{kouvaris2017evolution,szilagyi2020phenotypes}.

%TODO this needs a rewrite
The longstanding bidirectional exchange between evolutionary computation and evolutionary biology is truly remarkable, particularly in contrast with other areas of AI/ML where such efforts been sparser and more recent \citep{marblestone2016toward,richards2019deep}.
% synthesis of deep learning and neuroscience \citep{richards2019deep,marblestone2016toward}
%For instance, although deep learning is also rooted in a biological metaphor, there hasn't been as much fruitful impact translating neuroscience mechanisms to mainline research (although there have been efforts on certain frontiers CITE CITE CITE \citep{furber2014spinnaker}).
We look forward to seeing this exchange deepen and, in particular, fulfill concrete objectives in better explaining evolutionary computation, diagnosing failure cases, and prescribing appropriate methods for challenging domain problems.
