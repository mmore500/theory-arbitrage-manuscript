% or even be used to optimize biological organisms through directed evolution inspired by evolutionary computaiton approaches \citep{lalejini2022artificial}

\section{Conclusion} \label{sec:conclusion}

Sophisticated capabilities of the natural world --- still, in many cases, thoroughly unrivaled by artificial engineering --- strongly evidence the creative potential of evolutionary computation, and also lend it peculiar charisma.
% The notion of being seduced by this charisma is a reasonable criticism of evolutionary computation \citep{moore2023evolution,woodward2009evolution}.
The evolutionary metaphor in genetic algorithms and genetic programming has proven highly productive, in some manner capturing inherent potential and, more simply, capturing research attention.
Despite reasonable criticisms of metaphor-driven algorithm development \citep{moore2023evolution,sorensen2015metaheuristics}, there is compelling potential to strengthen the foundations of evolutionary computation by instead leaning further into the evolution metaphor and engaging analytical methods and explanatory theoretical constructs developed in evolutionary biology and allied fields.

In this work, we have reviewed notable examples where reaching into the biological literature has shown potential to improve our visibility into application-oriented evolutionary computation.
Such research, much of it very recent, reflects just first steps.
Important extensions remain to be fleshed out and connected to \textit{bona fide} real-world use cases.
Further, reviewed work touches only a small portion of potentially fruitful directions for arbitrage of theory and analysis from biology to evolutionary computation.
In the previous section, we have highlighted a few additional possibilities.

Productive exchange between biology and evolutionary computation is a two-way street.
Already, digital organisms find use cases in experiments conducted in conjunction with \textit{in vivo} inquiry \citep{TODO}.
However, we believe that application-oriented genetic programming and genetic algorithms have special value to offer in the realm of genotype-phenotype maps and evolvability.
Owing to its artificial nature, genotype-phenotype maps --- much less, maps with good evolvabilty --- do not come baked into evolutionary computation \textit{a priori} \citep{kirschner1998evolvability}.
As such, EC research has invested significant effort in teasing apart how the properties of a genotype-phenotype map influence outcomes from adaptive evolution \citep{TODO}.
A particularly promising line of EC work has sprung up in investigating how to harness unsupervised learning (e.g., autoencoders, LLMs) to generate evolvable genotype-phenotype maps \citep{moreno2018understanding,bradley2024openelm,wittenberg2023denoising,MORECITES}.
Corresponding work in biology, however, is only fledgling.
Indeed, such work considering evolvability as an unsupervised learning process has notably already been driven forward through collaboration with evolutionary computation practitioners \citep{kouvaris2017evolution,szilagyi2020phenotypes}.

%TODO this needs a rewrite
The longstanding bidirectional exchange between evolutionary computation and evolutionary biology is truly remarkable, particularly relative to other areas of AI/Ml where such efforts have arisen only more recently \citep{marblestone2016toward,richards2019deep}.
% synthesis of deep learning and neuroscience \citep{richards2019deep,marblestone2016toward}
%For instance, although deep learning is also rooted in a biological metaphor, there hasn't been as much fruitful impact translating neuroscience mechanisms to mainline research (although there have been efforts on certain frontiers CITE CITE CITE \citep{furber2014spinnaker}).
We look forward to seeing this exchange deepen and, in particular, fulfill concrete objectives in better explaining how aspects of evolutionary computation work mechanistically, diagnosing failure cases, and prescribing appropriate methods for difficult domain problems.
