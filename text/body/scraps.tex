% Although scientific literature is somewhat less charismatic, and can be more obscure to become acquainted with,

% notably with respect to genotype-phenotype maps
% However, we do argue that there is great potential for theory arbitrage to be bidirectional, and we identify some areas where evolutionary algoritms could make important contributipns to the corpus of biological literature.

% Here, we highlight the value of a second tier of bio-inspiration beyond direct inspiration from natural systems themselves with a focus on understanding algorithms instead of enhancing them.
% This second tier, .
% Third way Complement the engineer
% Given the engineering/objective-driven bent of genetic programming, it should come as no surprise that genetic programming had been faster on the first impulse to uptake of ideas/mechanisms from evolutionary biology that can be incorporated into runtime algoritm dynamics.

% This concern has been widely acknolwedged as a key barrier impeding research in the field from systematically building up and one that has stymied its palatteability
% particularly for bio-inspired methods \citep{del2019bio}.
% % perceived structural deficiencies of evolutionary algorithm approaches.

% we should augment bio-inspiration from  with inspiration from theory/methods developed by biologists to describe evolutionary dynamics and provide causal explanations for their nature.

% Difficulty in developing a cohesive framework should come as no surprise

% Computational reduction
% In some cases, however, explanatory rigor has been gained by leaning \textit{further} into the biological metaphor, looking past algorithmic inspiration from the facts of evolutionary biology to consider the methods and theory through which it is studied and understood.

% Biology hosts extensive hosts well-developed, quantative, and informative methods and theory.
% Although less front-and-center than inspiration from biological mechanisms, genetic programming has a rich history of benefiting --- and contributing --- to questions around the \textit{study} of biology.
% There is much more that can and should be done, and it has an important role to play in progress toward longstanding goals in the field to enhance unifying theory and explanatory rigor.
% Here, we review notable applications of biological theory and methods to \textit{understand}/analyze evolutionary algorithms, as opposed to driving the evolutionary process through algorithms with biological metaphors.

% core of the ethos into genetic programming and genetic algorithms

% has become ingrained in stands apart from .

% Indeed, these algorithms stand apart from  how deeply the evolutionary metaphor has become ingrained into the ethos of the field, and the very vocabulary use to describe and conceptualize.
% This peculiar

% Bio-inspiration is ingrained

% Given the extent to which bio-inspiration is ingrained into the vocaublary and ethos of genetic programming and genetic algorithms, it can be easy to lose track of the fact that bio-inspiration as to become obscure.
% Bio-inspiration is not intrinsic to the core optimization objectives of genetic programming and genetic algorithms, yet remains deeply ingrained as an organizing framework and source of inspiration for research in the field.

% Although not necessarily a direct objective of engineering optimization \textit{per se}, the biological metaphor  on optimization via genetic programming and genetic algorithms.
% , although it is difficult to untangle the extent to which this is because the charisma captured genetic programming community.

% This framing has been productive source of inspiration, but it has also contributed to : .
% This is owing to the \textit{ad hoc} nature of bio-inspired approaches,
% Progress on this problem, has been achieved by leaning \textit{further} into bio-inspiration.
% Past questions of how biological systems themselves operate, we can also gain insight/value from the methods, theory, and software tools that have been developed among evolutionary biologists to study evolution.


% Opening thoughts --- how can this be done in a positive way?
% - Bio-inspiration is powerful! and cool!; algorithms use bio-inspired mechanisms
% - but the narrative, analogy-driven ethos also drives an undercurrent of inferiority complex within the evolutionary computaiton field  (the "second-best" way to do things)
% - researchers have long been concerned about the rigor of the field
% - By docking the ship to
% -



%- evolution is powerful, and this power is the major driving factor of evolutionary algorithms

% - the origination of the field, many fruitful past and ongoing research directions in the field, are inspired by analogy to nature
% - lack of theory is a major problem in the field
% - to some degree, mimicking nature is a \textit{de facto} theory where we think it should work or it does work because it works in nature
%- this makes it difficult to relate ideas to each other
%- this makes it difficult to choose an appropriate approach for a problem or to tune the problem
%- in order to make progress on this problem, we need to move beyond inspiration from mechanisms in evolution and the natural world to incorporating the tools, theories, and methods of existing science designed to study it.

% - however, compared to other AI/ML fields the bio-inspiration in evolutonary algorithms has had more staying power
% - indeed, algorithms inspired through analogy to nature have been pivotal in defining the course of the field
% - examples include: ....


% - indeed, incorporating intuition/narrative instead of mathematically-driven approaches has been a major criticism of the field
% - indeed, cross-cutting theory in genetic programming is somewhat limited (schema theory.....)

% This alignment had ansecond major advantage.
% It gives access to the scientific community revolving around evolution.
% Indeed, as with the NSF BEACON Center this has been a major institutional boon.
% It goves access to additional resources funding and personnel
% However, it is also a source of theory and methods

% We're quick to try out the intuition onto doodads and whizbang mechsnisms to augment the algorithms, but slow to over the analytical methods, theory, and tools to establish disgnostics and a principled explanation for what's going on.

% There is a cross between intentipnally porting things over and stumbling into the realization that X is important!


% original inspiration for the field --- argue that these haven't delved into the biological literature as much as they could have/should have
% \begin{itemize}
%   \item koza, recombination and crossover
%   \item island-based model
%   \item coevolution (of tests and test case, ???)
%   \item speciation
% \end{itemize}

% classic work
% \begin{itemize}
%   \item population size "The role of population size in rate of evolution in genetic programming"
%   \item schemata/building blocks
%   \item diversity maintenance (EcoEA, Lexicase)
%   \item plasticity
%   \item genotype-phenotype maps
% \end{itemize}

% recent work
% \begin{itemize}
%   \item ecology assembly theory \citep{dolson2024reachability}
%   \item phylogeny for prediction/analysis \citep{hernandez2022can,shahbandegan2022untangling}
%   \item alife data standards: interoperation with bioinformatics infrastructure \citep{lalejini2019data,moreno2024apc}
%   \item reconstruction-based tracking \citep{moreno2022hstrat,moreno2024ecology}
% \end{itemize}

Genetic programming stands apart from other fields in AI/ML in the depth, and staying power, of bio-inspiration.
Indeed, one criticism of the field is that the evolution metaphor developed into somewhat as an end in and of itself rather than as a means to an end.

Nonetheless, it is certainly the case that ongoing extensions of the biological metaphor have been exceptionally productive.
To name a few, X, Y, Z (which we review in detail later)

This has been an uneven process of grabbing ideas from biology directly and trying stuff out and after the fact realizing that it relates to biology and is useful/important.
This empiricism-first, descriptive paradigm is like biology. \citep{welch2017wrong}.
Indeed, this approach has been a major criticism of biology --- "just-so" stories in evolutionary biology \citep{smith2016explanations}.
Both fields have dealt with an inferiority complex with fields that are more mathematically modelable/driven --- genetic programming to deep learning and evolutionary biology to physics.

(Although this is not always the case, cite experimental evolution and objective-driven evolution).
The enthusiasm for biological concepts has left genetic programming blind to the more descriptive/analytical aspects of evolutionary biology.

CLOSING SENTENCE
- there has been some reasonable criticism about whether this is the case because of an infatuation driving the productivity or the productivity leading to the
- the charisma/power of evolution is what motivated founding work in genetic algorithms/genetic programs
- evolution is also charismatic --- have evolutionary algorithms been hijacked?
Without the counterfactual, it's difficult to say whether adherence to the evolution metaphor is due to its productivity or the productivity has instead been driven by the infatuation/charisma of evolution.
